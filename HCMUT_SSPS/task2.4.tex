\subsection{Task 2.4: Design User Interface}
    Liên kết đến thiết kế trong Figma: \href{https://www.figma.com/file/iKigk9Qx8GSMj3kLY0gqzQ/Nhom7_report_3?type=design&node-id=0%3A1&mode=design&t=Ja6DS4vqBqHGVIOn-1}{Nhom7\_report\_3}
    \subsubsection{Giao diện chung}
    \subsubsubsection{Giao diện trang chủ}
    \begin{center}
    \begin{figure}[!htp]
    \begin{center}
     \includegraphics[scale=.32]{images/Task 2.4/TrangChu.png}
    \end{center}
    \label{refhinh1}
    \end{figure}
    \end{center}
    \textbf{Mô tả:}
    \begin{itemize}
    \item Đây là giao diện đầu tiên khi người dùng truy cập vào trang web.
    \item Người dùng click vào "Đăng nhập" để tiến hành đăng nhập và sử dụng các dịch vụ.
    \end{itemize}

    \newpage
    \subsubsubsection{Giao diện đăng nhập}
    \begin{center}
    \begin{figure}[!htp]
    \begin{center}
     \includegraphics[scale=.32]{images/Task 2.4/DangNhap.png}
    \end{center}
    \label{refhinh1}
    \end{figure}
    \end{center}

    \textbf{Mô tả:}
    \begin{itemize}
        \item Giao diện bao gồm khu vực nhập thông tin đăng nhập (tài khoản, mật khẩu) của người dùng và một số thông tin như lưu ý, thông tin hỗ trợ kỹ thuật và bản quyền.
        \item Khu vực nhập thông tin bao gồm khung nhập thông tin tài khoản, mật khẩu, nút chọn đăng nhập và nút xóa thông tin đã nhập bên trên. Ngoài ra còn có checkbox để người dùng chọn cảnh báo khi có người đăng nhập vào trang web khác.
        \item Bên cạnh đó, khi người dùng muốn thay đổi mật khẩu hay họ quên mật khẩu và cần lấy lại thì có thể chọn để chuyển hướng đến trang đổi mật khẩu và quên mật khẩu.
    \end{itemize}
    \newpage
    \subsubsubsection{Giao diện thay đổi mật khẩu}
    \begin{center}
    \begin{figure}[!htp]
    \begin{center}
     \includegraphics[scale=.32]{images/Task 2.4/ChangePassword.png}
    \end{center}
    \label{refhinh1}
    \end{figure}
    \end{center}
    
    \textbf{Mô tả:}
    \begin{itemize}
        \item Giao diện bao gồm khu vực nhập thông tin để thay đổi mật khẩu bao gồm username, old password, new password và confirm new password.
        \item Sau khi bấm chọn submit thì hệ thống sẽ xác nhận thông tin và thay đổi mật khẩu cho người dùng. Nếu thông tin sai thì sẽ hiển thị thông báo để người dùng nhập lại.
        \item Bên cạnh đó, ở phần đầu trang còn có khu vực để người dùng có thể chuyển sang giao diện đặt lại mật khẩu.
    \end{itemize}
    \newpage
    \subsubsubsection{Giao diện đặt lại mật khẩu}
    \begin{center}
    \begin{figure}[!htp]
    \begin{center}
     \includegraphics[scale=.32]{images/Task 2.4/ResetPassword.png}
    \end{center}
    \label{refhinh1}
    \end{figure}
    \end{center}

    \textbf{Mô tả:}
    \begin{itemize}
        \item Giao diện bao gồm khu vực nhập thông tin để đặt lại mật khẩu bao gồm username, và email.
        \item Sau khi bấm chọn submit thì hệ thống sẽ xác nhận thông tin và gửi "link" xác thực việc đặt lại mật khẩu thông qua email mà người dùng cung cấp. Người dùng cần truy cập vào "link" này và hoàn thành các bước tiếp theo để đặt lại mật khẩu.
        \item Bên cạnh đó, ở phần đầu trang còn có khu vực để người dùng có thể chuyển sang giao diện thay đổi mật khẩu.
    \end{itemize}
    \newpage
    \subsubsection{Giao diện của sinh viên}
    \subsubsubsection{Giao diện dịch vụ của tôi}
    \begin{center}
    \begin{figure}[!htp]
    \begin{center}
     \includegraphics[scale=.32]{images/Task 2.4/DichVuCuaToi(SV).png}
    \end{center}
    \label{refhinh1}
    \end{figure}
    \end{center}

    \textbf{Mô tả:}
    \begin{itemize}
    \item Khi người dùng click chọn vào "DỊCH VỤ CỦA TÔI", danh sách các dịch vụ của từng đối tượng sẽ được hiển thị tại đây.
    \item Nếu là sinh viên, danh sách dịch vụ sẽ gồm: ĐĂNG KÍ IN TÀI LIỆU, MUA THÊM TRANG IN, NHẬT KÍ SỬ DỤNG DỊCH VỤ IN.
    \end{itemize}
    
    \newpage
    \subsubsubsection{Giao diện đăng kí in tài liệu}
    \begin{center}
    \begin{figure}[!htp]
    \begin{center}
     \includegraphics[scale=.32]{images/Task 2.4/DangKiIn.png}
    \end{center}
    \label{refhinh1}
    \end{figure}
    \end{center}
    \textbf{Mô tả:}
    \begin{itemize}
        \item Giao diện bao gồm chọn tập tin, tùy chọn thuộc tính, chọn máy in và nút xác nhận đăng ký.
        \item Sinh viên chỉ được tải lên một file duy nhất, hệ thống sẽ tự động kiểm tra tính hợp lệ của file và thông báo lại cho sinh viên.
        \item Nút "Tùy chọn thuộc tính" chỉ khả dụng sau khi file hợp lệ được tải lên, sinh viên có thể chọn máy in trước khi tải file.
        \item Sinh viên có thể không cần tùy chọn thuộc tính và các thuộc tính sẽ được đặt mặc định.
        \item Sau Khi file hợp lệ được tải lên và máy in được chọn sinh viên phải nhấn nút "Đăng ký" để tiến hành in. Nút "Đăng ký" sẽ không khả dụng cho tới sinh viên thực hiện đủ hai điều kiện nêu trên.
    \end{itemize}
    \newpage
    \subsubsubsection{Giao diện tuỳ chọn thuộc tính in}
    \begin{center}
    \begin{figure}[!htp]
    \begin{center}
     \includegraphics[scale=.32]{images/Task 2.4/TuyChonThuocTinh.png}
    \end{center}
    \label{refhinh1}
    \end{figure}
    \end{center}
    \textbf{Mô tả:}
    \begin{itemize}
        \item Giao diện bao gồm các thuộc tính như khổ giấy, trang in, mặt in và số lượng in. Ngoài ra còn có phần preview các trang ở bên trái để sinh viên có cái nhìn tổng quát.
        \item Tất cả các thuộc tính đều được đặt mặc định sẵn như trong hình.
        \item Sau khi cài đặt các thuộc tính xong sinh viên cần nhấn nút "Hoàn tất" để lưu lại cài đặt.
        \item Nút "Đặt về mặc định" cấu hình lại các thuộc tính về mặc định đề phòng trường hợp sinh viên chỉnh các thuộc tính không ưng ý. 
    \end{itemize}
    
    \newpage
    \subsubsubsection{Giao diện mua trang in}
    \begin{center}
    \begin{figure}[!htp]
    \begin{center}
     \includegraphics[scale=.32]{images/Task 2.4/MuaTrangIn.png}
    \end{center}
    \label{refhinh1}
    \end{figure}
    \end{center}

    \textbf{Mô tả:}
    \begin{itemize}
    \item Hệ thống cung cấp thông tin về số trang in hiện tại của người dùng (theo khổ A4).
    \item Khi muốn mua thêm trang trang in, sinh viên nhập số lượng cần mua và bấm nút "Đăng ký" để tạo đơn hàng.
    \item Trường "Lịch sử đăng ký" sẽ cung cấp thông tin về tất cả các đăng ký mua trang in của sinh viên. Ở cột "Thanh toán", có 2 giá trị là "Đã thanh toán" và "Thanh toán ngay" (nếu sinh viên chưa thanh toán).
    \end{itemize}

    \newpage
    \subsubsubsection{Giao diện chọn hình thức thanh toán}
    \begin{center}
    \begin{figure}[!htp]
    \begin{center}
     \includegraphics[scale=.32]{images/Task 2.4/ThanhToan.png}
    \end{center}
    \label{refhinh1}
    \end{figure}
    \end{center}

    \textbf{Mô tả:}
    \begin{itemize}
    \item Từ giao diện mua trang in, để thanh toán đơn hàng, sinh viên click vào "Thanh toán ngay".
    \item Khi đó, một của sổ "Chọn hình thức thanh toán" sẽ bật lên. Sinh viên cần lựa chọn hình thức thanh toán phù hợp.
    \begin{itemize}
        \item Sinh viên có thể chọn "Quay lại" nếu không muốn thực hiện thanh toán.
        \item Khi sinh viên nhấn "Tiếp tục" thì trang web sẽ chuyển sang trang cổng thanh toán của các bên dịch vụ tương ứng. Tại đó, sinh viên sẽ tiếp tục các bước thanh toán theo yêu cầu của cổng thanh toán.
    \end{itemize}
    \item Sau khi hoàn tất thanh toán, giao diện sẽ cập nhật trang thái từ "Thanh toán ngay" thành "Đã thanh toán" và "Số trang in hiện tại" cũng sẽ được cập nhật.
    \end{itemize}


    \newpage
    \subsubsection{Giao diện của SPSO}
    \subsubsubsection{Giao diện dịch vụ của tôi}
    \begin{center}
    \begin{figure}[!htp]
    \begin{center}
     \includegraphics[scale=.32]{images/Task 2.4/DichVuCuaToi(SPSO).png}
    \end{center}
    \label{refhinh1}
    \end{figure}
    \end{center}

    \textbf{Mô tả:}
    \begin{itemize}
    \item Khi người dùng click chọn vào "DỊCH VỤ CỦA TÔI", danh sách các dịch vụ của từng đối tượng sẽ được hiển thị tại đây.
    \item Nếu là SPSO, danh sách dịch vụ sẽ gồm: QUẢN LÝ CÁC MÁY IN, CẤU HÌNH HỆ THỐNG, NHẬT KÍ SỬ DỤNG DỊCH VỤ IN CỦA SINH VIÊN, CÁC BÁO CÁO VỀ VIỆC SỬ DỤNG DỊCH VỤ IN.
    \end{itemize}

    \newpage
    \subsubsubsection{Giao diện thêm máy in mới}
    \begin{center}
    \begin{figure}[!htp]
    \begin{center}
     \includegraphics[scale=.32]{images/Task 2.4/ThemMayIn.png}
    \end{center}
    \label{refhinh1}
    \end{figure}
    \end{center}

    \textbf{Mô tả:}
    SPSO sẽ chọn cơ sở đào tạo và tòa nhà cần lắp đặt thêm máy in. Sau đó, SPSO sẽ nhập ID cho máy in mới
    \begin{itemize}
        \item Nếu ID hợp lệ thì ID sẽ chuyển sang màu xanh và SPSO có thể tiếp tục nhập Tên, Mẫu mã và các ghi chú đính kèm. Sau đó, SPSO có thể nhấn xác nhận để thêm máy in vào hệ thống.
        \item Nếu ID không hợp lệ, ID sẽ chuyển sang màu đỏ và SPSO sẽ cần phải nhập lại ID
    \end{itemize} 
    
    \newpage
    \subsubsubsection{Giao diện quản lí máy in}
    \begin{center}
    \begin{figure}[!htp]
    \begin{center}
     \includegraphics[scale=.32]{images/Task 2.4/QuanLyMayIn.png}
    \end{center}
    \label{refhinh1}
    \end{figure}
    \end{center}

    \textbf{Mô tả:}
    SPSO có thể lựa chọn cơ sở đào tạo, tòa nhà và máy in cụ thể để hiện ra các thông tin hiện tại của máy in đó. Nếu muốn thay đổi trạng thái hoạt động của máy in đó, SPSO có thể nhấn vào nút (Thay đổi) để hiện lên pop-up. Khi đó, pop-up sẽ yêu cầu SPSO nhập ID để xác nhận sau đó chọn bật hoặc tắt theo nhu cầu.

    \newpage
    \subsubsubsection{Giao diện cấu hình hệ thống}
    \begin{center}
    \begin{figure}[!htp]
    \begin{center}
     \includegraphics[scale=.32]{images/Task 2.4/CauHinhHeThong.png}
    \end{center}
    \label{refhinh1}
    \end{figure}
    \end{center}

    \textbf{Mô tả:}
    Theo như các diagram trước đó về configure system, 3 chức năng chính được chia theo 3 nhánh hoạt động riêng. Tuy nhiên trong quá trình thiết kế và hiện thực, việc chia chức năng ra như vậy là không cần thiết và tương đối lãng phí không gian trong giao diện. Vì vậy, một quyết định được đưa ra đó là gộp 3 chức năng đó vào trong 1 trang duy nhất. Khi nhấn lưu thì tất cả thông số mặc định sẽ đồng thời được kiểm tra và cập nhật.\\